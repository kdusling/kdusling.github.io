\documentclass[multi,preview,varwidth=false,border=5,12pt]{standalone}
\newcounter{Qnum}

\usepackage{assignments}
\standaloneenv{question}

\excludecomment{solution}\let\endsolution\relax

\begin{document}

\begin{question}
Convert 2250 millimeters to meters.

\begin{solution}
  \begin{align*}
      \left(2250~\mm\right)\times\left(\frac{1~\m}{1000~\mm}\right)=2.25~\m
  \end{align*}
\end{solution}

\end{question}

\begin{question}
Convert a length of 24 feet to meters.

\begin{solution}
  \begin{align*}
      \left(24~\ft\right)\times\left(\frac{1~\m}{3.281~\ft}\right)=7.3~\m
  \end{align*}
\end{solution}
\end{question}

\begin{question}
Convert a distance of 5 km into miles.

\begin{solution}
  \begin{align*}
      \left(5~\km\right)\times\left(\frac{1~\mi}{1.609~\km}\right)=3.1~\mi
  \end{align*}
\end{solution}
\end{question}

\begin{question}
Convert $4.45\times 10^6$ square millimeters to square meters.

\begin{solution}
  \begin{align*}
      \left(4.45\times 10^6~\mm^2\right)\times\left(\frac{1~\m}{1000~\mm}\right)^2=4.45~\m^2
  \end{align*}
\end{solution}

\end{question}

\begin{question}
Convert a volume of 14 liters to cubic meters.

\begin{solution}
  \begin{align*}
      \left(14~\rm{L}\right)\times\left(\frac{1~\m^3}{1000~\rm{L}}\right)=0.014~\m^3
  \end{align*}
\end{solution}
\end{question}

\begin{question}
Convert a volume of 75 gallons to cubic feet.

\begin{solution}
  \begin{align*}
      \left(75~\rm{gal}\right)\times\left(\frac{1~\ft^3}{7.48~\rm{gal}}\right)=10~\ft^3
  \end{align*}
\end{solution}
\end{question}

\begin{question}
A car is moving at 88~mph.  What is its speed when expressed in ft/$s$?

\begin{solution}
  \begin{align*}
      \left(88~\rm{mph}\right)=88\left(\frac{\mi}{\hr}\right)\times\left(\frac{1~\hr}{3600~s}\right)\times\left(\frac{5280~\ft}{1~\mi}\right)=129~\ft/s
  \end{align*}
\end{solution}
\end{question}

\begin{question}
The fastest train in Italy is the Frecciarossa having a top speed of 360 km/h.  \newline
a) Express this speed in meters per second.  \newline
b) Express this speed in mph.

\begin{solution}
  \begin{align*}
      \left(360~\rm{kmph}\right)=360\left(\frac{\km}{\hr}\right)\times\left(\frac{1~\hr}{3600~s}\right)\times\left(\frac{1000~\m}{1~\km}\right)=100~\m/s
  \end{align*}

  \begin{align*}
      \left(360~\rm{kmph}\right)=360\left(\frac{\km}{\hr}\right)\times\left(\frac{1~\mi}{1.609~\km}\right)=224~\rm{mph}
  \end{align*}

\end{solution}
\end{question}

\begin{question}
A car travelling at 60 mph has a kinetic energy of 190,000 ft-lbs.\newline
Compute the weight of the car in pounds.

\begin{solution}
  \begin{align*}
      60~\rm{mph}=88~\ft/s
  \end{align*}

  \begin{align*}
      m=\frac{2\rm{KE}}{v^2}=\frac{2\times 190,000~\ft\cdot\lb}{\left(88 ~\ft/s\right)^2}=49.07~\frac{\lb\cdot s^2}{\ft}=49.07~\rm{slug}
  \end{align*}

    \begin{align*}
    \rm{weight}=mg=49.07~\rm{slug}\times 32.2 \frac{\ft}{s^2}=1580~\lb
  \end{align*}

\end{solution}
\end{question}

\begin{question}
What is the weight of a gallon of water if it has a mass of 0.26 slug?

\begin{solution}
  \begin{align*}
      w=mg=0.26~\slug \times 32.2~\ft/s^2=8.4~\lb
  \end{align*}
\end{solution}

\end{question}

\begin{question}
An Olympic weight plate has a mass of 20 kg.  What is its weight in pounds?

\begin{solution}
  \begin{align*}
      w=mg=20~\kg \times 9.81~\m/s^2=196.2~\N\times \left(\frac{1~\lb}{4.448~\N}\right)=44.1~\lb
  \end{align*}
\end{solution}

\end{question}

\begin{question}
Using a bathroom scale a man measures his weight to be 172 lb (pounds-force).  \newline
a) What is his mass in lbm (pounds-mass)?  \newline
b) What is his mass in slugs? \newline
c) What is his mass is kg? \newline
d) What is his weight in N?

\begin{solution}
  \begin{align*}
      &a)~ 172~\lbm\\
      &b)~ m=w/g=172~\lb/(32.2~\ft/s^2)=5.34~\slug\\
      &c)~ 5.34~\slug\times \left(\frac{14.59~\kg}{1~\slug}\right)=78~\kg\\
      &d)~78~\kg \times 9.81 ~\m/s^2=765~\N
  \end{align*}
\end{solution}

\end{question}

\begin{question}
The service manual for your car specifies that lug nuts should be tightened to a torque of 100 ft-lbs.  What is this torque in units of N-m?

\begin{solution}
  \begin{align*}
    100~\ft\cdot\lb \times\left(\frac{4.448~\N}{\lb}\right)\times\left(\frac{\m}{3.281~\ft}\right)=135.6~\N\cdot\m
  \end{align*}
\end{solution}

\end{question}

\begin{question}
A lockring for a bicycle cassette should be tightened to a torque of 40 N-m.  What is this torque in units of in-lbs?

\begin{solution}
  \begin{align*}
    40~\N\cdot\m \times\left(\frac{\lb}{4.448~\N}\right)\times\left(\frac{39.37~\inch}{\m}\right)=354~\inch\cdot \lb
  \end{align*}
\end{solution}
\end{question}

\end{document}
