\documentclass[multi,preview,varwidth=false,border=5,12pt]{standalone}
%\documentclass[12pt]{article}

\newcounter{Qnum}
\usepackage{assignments}
\standaloneenv{question}


\excludecomment{solution}\let\endsolution\relax


\begin{document}


\begin{center}
\section*{Nature of fluids}
\end{center}

\subsection*{Density, specific weight \& specific gravity}

\begin{question}
A test mass used by the LISA Pathfinder mission is a cube of solid gold--platinum alloy, measuring 4.6 cm on a side and weighing 1.96 kg.  Compute the cube's density, specific weight and specific gravity.

\begin{solution}
  We start by computing the cube's density which is its mass divided by its volume.  Since we want our answer to be in units of $\kg/\m^3$ we first convert the length of the side from 4.6 cm to 0.046 m.  The volume is therefore $V=\left(0.046~\m\right)^3=9.73\times 10^{-5}~\m^3$.
  \begin{align*}
  \rho=\frac{m}{V}=\frac{1.96~\kg}{9.73\times 10^{-5}~\m^3}=20,144~\kg/\m^3
  \end{align*}
  This is a reasonable result as it falls between the densities of platinum and gold which are $21,450~\kg/\m^3$ and  $19,300~\kg/\m^3$ respectively. [https://physics.info/density/]

  The specific weight is then computed from
  \begin{align*}
  \gamma=\rho g=20,144~\kg/\m^3\times 9.81~\m/s^2=1.976\times 10^5~\N/\m^3
  \end{align*}

  Finally, the specific gravity can be computed as
  \begin{align*}
  sg=\frac{1.976\times 10^5~\N/\m^3}{9.81\times 10^3~\N/\m^3}=20.144
  \end{align*}
\end{solution}
\end{question}


\begin{question}
Air at $40\C$ and standard atmospheric pressure has a specific weight of 11.05 N/m$^3$.
Calculate its density.

\begin{solution}
  We can use the relation $\gamma=\rho g$ to find the density from the specific weight.
  \begin{align*}
      \rho=\frac{\gamma}{g}=\frac{11.05~\N/\m^3}{9.81~\m/s^2}=1.126~\kg/\m^3
  \end{align*}
\end{solution}

\end{question}


\begin{question}
A storage vessel for gasoline (sg=0.68) is a vertical cylinder 10 m in diameter.  If it is filled to a depth of 6.75 m, calculate the weight and mass of the gasoline.

\begin{solution}
  This is a two step problem.  First we will use the specific gravity to compute the density and specific weight.

  \begin{align*}
    \rho_{\rm gas}&=sg\times \rho_{\rm water}=0.68\times 1000~\kg/\m^3=680~\kg/\m^3\\
    \gamma_{\rm gas}&=sg\times \gamma_{\rm water}=0.68\times 9.81~\kN/\m^3=6.67~\kN/\m^3
  \end{align*}

  Then using the volume of gasoline we will get the total weight and mass.  The volume of gasoline is the volume of the cylinder as described.

  \begin{align*}
      V={\rm Height}\times {\rm Area}=6.75~\m\times \left(\frac{\pi (10~\m)^2}{4}\right)=530~\m^3
  \end{align*}

  The mass and weight is therefore

  \begin{align*}
      m&=\rho\times V=680~\frac{\kg}{\m^3}\times 530~\m^3 = 3.6\times 10^{5}~\kg\\
      w&=\gamma\times V=6.67~\frac{\kN}{\m^3}\times 530~\m^3 = 3500~\kN
  \end{align*}

\end{solution}

\end{question}

\begin{question}
A storage vessel for gasoline (sg=0.68) is a vertical cylinder 30~ft in diameter.  If it is filled to a depth of 22~ft, calculate the number of gallons and weight of the gasoline.

\begin{solution}
  This question is similar to the one before except now that it asks for the weight of the gasoline (in pounds) and the volume (in gallons).  The specific weight of the gasoline is

  \begin{align*}
    \gamma_{\rm gas}=sg\times \gamma_{\rm water}=0.68\times 62.4~\lb/\ft^3=42.4~\lb/\ft^3
  \end{align*}

  Then using the volume of gasoline we will get the total weight.  The volume of gasoline is the volume of the cylinder as described.

  \begin{align*}
      V={\rm Height}\times {\rm Area}=22~\ft\times \left(\frac{\pi (30~\ft)^2}{4}\right)=15550~\ft^3
  \end{align*}

  The weight is therefore

  \begin{align*}
      w=\gamma\times V=42.4~\frac{\lb}{\ft^3}\times 15550~\ft^3 = 6.60\times 10^5~\lb
  \end{align*}

  The volume in gallons is

    \begin{align*}
      V==15550~\cancel{\ft^3}\times\left(\frac{7.48~\gal}{\cancel{\ft^3}}\right)=1.16\times 10^5~\gal
  \end{align*}

\end{solution}

\end{question}


\begin{question}
Liquid ammonia has a specific gravity of 0.826.  Calculate the volume in cm$^3$  that would weigh 5.0 lb.

\begin{solution}
 2745~cm$^3$
\end{solution}

\end{question}

\begin{question}
What is the specific gravity of 38$^\circ$ API oil?

\end{question}

\subsection*{Pressure}

\begin{question}
A hydraulic press that must exert a force of 4000~lbs operates with a 2~in diameter cylinder.  Compute the required oil pressure.

\begin{solution}
1273~psi
\end{solution}


\end{question}

\begin{question}
The maximum pressure a fluid power cylinder can sustain is 25.0 MPa.  Compute the minimum diameter necessary for the piston to exert a force of 50~kN.

\begin{solution}
50.5~mm
\end{solution}

\end{question}

\subsection*{Compressibility}

\begin{question}
A hydraulic system operates using machine oil having a bulk modulus $K=189,000~\psi$.  What is the percentage change in volume as the system pressure is increased from zero to 4000 psi?

\begin{solution}
-2.1 percent
\end{solution}

\end{question}

\begin{question}
A hydraulic cylinder filled with water has an inside diameter of 1.0~in and a length of 2.0~ft. How many pounds of force must be applied to a piston at the end of the cylinder to compress the water by 0.25~in?

\begin{solution}
2585~lbs
\end{solution}

\end{question}

\subsection*{Viscosity}

\begin{question}
Convert a dynamic viscosity measurement of 2500~cP into $\Pa\cdot \s$.

\begin{solution}
$2.5~\Pa\cdot \s$
\end{solution}
\end{question}


\begin{question}

 Estimate the shear viscosity (in centipoise) of castor oil using the experimental viscometer data shown in the figure below.

 \includegraphics[width=4in]{imgs/CastorOil.png}

\begin{solution}
231 cP
\end{solution}
\end{question}

\begin{question}

The following three questions are based on the experimental viscometer data for French dressing shown in the figure below.

 \includegraphics[width=4in]{imgs/FrenchDressing.png}

What best describes the viscous behavior of French dressing?

\begin{enumerate}
  \item Bingham
  \item Dilatant (shear thickening)
  \item Newtonian
  \item Pseudoplastic (shear thinning)
\end{enumerate}

\begin{solution}
Pseudoplastic (shear thinning)
\end{solution}

\end{question}


\begin{question}

 Estimate the apparent viscosity of French dressing (in centipoise) at a shear rate of $\dot{\gamma}=10 s^{-1}$.

\begin{solution}
458 cP
\end{solution}

\end{question}

\begin{question}

 Estimate the apparent viscosity of French dressing (in centipoise) at a shear rate of $\dot{\gamma}=40 s^{-1}$.

\begin{solution}
227 cP
\end{solution}

\end{question}

\end{document}
