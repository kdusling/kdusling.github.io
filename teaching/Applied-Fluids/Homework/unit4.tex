\documentclass[multi,preview,varwidth=false,border=5,12pt]{standalone}
%\documentclass[12pt]{article}

\newcounter{Qnum}
\usepackage{assignments}
\standaloneenv{question}


\excludecomment{solution}\let\endsolution\relax


\begin{document}

\begin{center}
\section*{Buoyancy and Stability}
\end{center}

\begin{question}

A bowling alley owns a collection of bowling balls having a circumference of 27 inches and composed of a graphite core (density=$2.266~\textrm{g}/\textrm{cm}^3$).  The weights available include 8, 10, 14 and 16 pounds.  Which balls, if any, float in fresh water?

\begin{enumerate}
  \item Only the 8 pound floats.
  \item The 8 and 10 pound float.
  \item The 8, 10 and 14 pound float.
  \item They all float.
  \item They all sink.
\end{enumerate}

\begin{solution}
    The 8 and 10 pound float.
\end{solution}

\end{question}


\begin{question}

A 2~m long standard steel pipe has an outside diameter of 168~mm and weighs 554~N.  Will the pipe float or sink in glycerin (sg=1.26) if its ends are closed?  What force would be required to hold it in equilibrium?

\begin{enumerate}
  \item Sinks, $F_{\rm ext}=6$~N
  \item Floats, $F_{\rm ext}=0$~N
  \item Floats, $F_{\rm ext}=6$~N
  \item Sinks, $F_{\rm ext}=12$~N
  \item Sinks, $F_{\rm ext}=3$~N
\end{enumerate}

\begin{solution}
    Sinks, $F_{\rm ext}=6$~N
\end{solution}

\end{question}



\begin{question}

A helium balloon is filled with 0.5 cu. ft. of helium gas.  If the empty balloon weighs 5 grams what is the minimum weight required to hold down the balloon.  Take the specific weight of the helium gas and air to be 0.0103~$\lb/\ft^3$ and 0.075~$\lb/\ft^3$ respectively.

\begin{enumerate}
  \item 2~g
  \item 10~g
  \item 12~g
  \item 15~g
  \item 17~g
\end{enumerate}

\begin{solution}
    10~g
\end{solution}

\end{question}


\begin{question}

A container for an emergency beacon is a rectangle 30.0~in wide, 40.0~in long, and 22.0 in high.  Its center of gravity is 10.50~in above its base.  The container weights 300~lb.  Will the box be stable with the $30\times 40$~in side parallel to the surface of plain water?  Report the distance from the base to the metacenter.

\begin{enumerate}
  \item $y_{\rm mc}=10.8$~in (stable)
  \item $y_{\rm mc}=14.3$~in (stable)
  \item $y_{\rm mc}=14.3$~in (unstable)
  \item $y_{\rm mc}=16.0$~in (stable)
  \item $y_{\rm mc}=16.0$~in (unstable)
\end{enumerate}

\begin{solution}

    \begin{align*}
    y_{cb}=\frac{12^3}{2(62.4)}\frac{W}{wl}=0.0115385W\nonumber\\
    MB=\frac{62.4}{12^4}\frac{lw^3}{W}=3250/W
    \end{align*}

    $y_{\rm mc}=14.3$~in (stable)

\end{solution}

\end{question}


\begin{question}

The following four questions consider a fresh-water buoy that is in the shape of a solid cylinder.  It has an $18~\inch$ diameter, is $4~\ft$ long and is constructed from EVA foam of density 931 $\kg/\m^3$.

If the buoy floats upright, how much of its length is above the water's surface?
Report your result in inches.

\begin{solution}
    $3.3~\inch$
\end{solution}

\end{question}

\begin{question}
Compute the metacentric radius (MB) for the buoy.  Report your result in inches.

\begin{solution}
    $0.45~\inch$
\end{solution}

\end{question}

\begin{question}
What is the distance from the bottom of the buoy to the position of the metacenter (mc)?  Report your result in inches.

\begin{solution}
    $22.8~\inch$
\end{solution}

\end{question}

\begin{question}
Is the buoy stable?

\begin{solution}
    yes.
\end{solution}

\end{question}



\end{document}
